\documentclass[article]{beamer}% para se tiver muitas sections
%\documentclass[11pt,compress,xcolor=dvipsnames]{beamer}
%---------------------------------------------------------------------
% Color and themes
%---------------------------------------------------------------------
\definecolor{ipb}{rgb}{0.36, 0.54,0.66}
%\definecolor{royalazure}{rgb}{0.07, 0.04, 0.56}
\definecolor{royalazure}{RGB}{8,71,155}
\setbeamercolor{footline}{bg=ipb}
\setbeamercolor{frametitle}{bg=ipb,fg=white}
\setbeamercolor{title}{bg=ipb}
\setbeamerfont{frametitle}{size=\large}
\setbeamertemplate{bibliography item}[text]
\setbeamertemplate{caption}[numbered]
\setbeamertemplate{blocks}[rounded][shadow]
\setbeamercolor{palette primary}{use=structure,fg=white,bg=structure.fg}
\setbeamercolor{palette secondary}{use=structure,fg=white,bg=structure.fg!75!black}
\setbeamercolor{palette tertiary}{use=structure,fg=white,bg=structure.fg!50!black}
\setbeamercolor{palette quaternary}{fg=white,bg=structure.fg!50!black}
\setbeamercolor*{sidebar}{use=structure,bg=structure.fg}
\setbeamercolor{titlelike}{parent=palette primary}
\setbeamercolor{block title}{bg=ipb,fg=white}
\setbeamercolor*{block title example}{use={normal text,example text},bg=white,fg=ipb}
\setbeamercolor{fine separation line}{}
\setbeamercolor{item projected}{fg=black}
\setbeamercolor{palette sidebar primary}{use=normal text,fg=normal text.fg}
\setbeamercolor{palette sidebar quaternary}{use=structure,fg=structure.fg}
\setbeamercolor{palette sidebar secondary}{use=structure,fg=structure.fg}
\setbeamercolor{palette sidebar tertiary}{use=normal text,fg=normal text.fg}
\setbeamercolor{palette sidebar quaternary}{fg=ipb}
%\setbeamercolor{section in sidebar}{fg=brown}
%\setbeamercolor{section in sidebar shaded}{fg=grey}
\setbeamercolor{sidebar}{bg=ipb}
\setbeamercolor{sidebar}{parent=palette primary}
\setbeamercolor{structure}{fg=ipb}
%\setbeamercolor{subsection in sidebar}{fg=brown}
%\setbeamercolor{subsection in sidebar shaded}{fg=grey}
\setbeamercolor{section in head/foot}{fg=white,bg=royalazure}
%\setbeamercolor{subsection in head/foot}{fg=white,bg=royalazure}
\usetheme{Warsaw}
%---------------------------------------------------------------------
% Footline
%---------------------------------------------------------------------
\setbeamertemplate{footline}
 {\leavevmode
\hbox{
   \begin{beamercolorbox}[wd=0.49\paperwidth,ht=2.25ex,dp=1ex,leftskip=0.3cm]{author in head/foot}%
     \usebeamerfont{author in head/foot}\textcolor{white}{\insertshortauthor}
   \end{beamercolorbox}%
   \begin{beamercolorbox}[wd=0.34\paperwidth,ht=2.25ex,dp=1ex,center]{title in head/foot}
     \usebeamerfont{title in head/foot}\textcolor{white}{\insertshorttitle}
   \end{beamercolorbox}%
   \begin{beamercolorbox}[wd=0.17\paperwidth,ht=2.25ex,dp=1ex,leftskip=0.3cm,rightskip=0.3cm]{title in head/foot}%
   \hfill\usebeamerfont{page number in head/foot}
   \insertframenumber{} / \textcolor{white}{\inserttotalframenumber}
   \end{beamercolorbox}}
}
%---------------------------------------------------------------------
% Packages
%---------------------------------------------------------------------
\usefonttheme[]{serif}
\usepackage{amsmath, latexsym, color, graphicx, amssymb, bm, here}
\usepackage{epsf, epsfig, pifont,tikz,subfigure}
\usepackage{graphics, calrsfs}
\usepackage{times}
\usepackage{fancybox,calc}
\usepackage{palatino,mathpazo}
\usepackage{amsfonts}
\usepackage{wrapfig}
\usepackage{multicol}
\usepackage{sidecap}
\usepackage{academicons}
%\usepackage{pdfauthor}
%\usepackage{pdfcreator}
\usepackage{hyperref}
\usepackage{listings}
\usepackage[brazil]{babel}
\usepackage{fourier}
\usepackage{arabtex}
\usepackage{utf8}
\setcode{utf8}
%---------------------------------------------------------------------
% Definir
%---------------------------------------------------------------------
\def\inst#1{\unskip$^{#1}$}
\def\orcidID#1{\unskip$^{[#1]}$}
\def\fnmsep{\unskip$^,$}
\def\email#1{{\tt#1}}


%---------------------------------------------------------------------
% Dados
%---------------------------------------------------------------------
\title{Monitoria Inicial de Álgebra Linear}
%opçao para 1 autor
\author{Luís Felipe Marques}
%opçao para 2 autor
%\author{Primeiro Autor~\orcidID{a12345} \and Segundo Autor~\orcidID{a12345}}
%opcao para 3 autor
%\author{Primeiro Autor~\orcidID{a12345}\\
%        Segundo Autor~\orcidID{a12345}\\
 %       Terceiro Autor~\orcidID{a12345}
  %      }
%\institute{Instituto Politécnico de Bragança- Escola Superior de Tecnologia e Gestão\\
            %\vspace{0.3cm}
            %Licenciatura em Curso}
\institute{Fundação Getulio Vargas - Escola de Matemática Aplicada\\
            \vspace{0.3cm}
            Ciência de Dados
            }
\date{\vfill\scriptsize{18 de agosto de 2023}\\\vspace{0.2cm}\includegraphics[scale=0.5]{Imagens/emaplogo.png}}
%---------------------------------------------------------------------
% Index
%---------------------------------------------------------------------
\AtBeginSection[]
{
  \begin{frame}{Conteúdo}
    \tableofcontents[currentsection]
  \end{frame}
}
\definecolor{azuel}{rgb}{0.07, 0.04, 0.56}
	\definecolor{backcolour}{rgb}{0.95,0.95,0.92}
	\definecolor{codegreen}{rgb}{0,0.6,0}
	\definecolor{mygreen}{RGB}{28,172,0}
	\definecolor{mylilas}{RGB}{170,55,241}
	\definecolor{codegray}{rgb}{0.5,0.5,0.5}
	\definecolor{codepurple}{rgb}{0.58,0,0.82}
	\lstdefinestyle{list2}{language=Matlab,% lst
		basicstyle=\color{black},
		backgroundcolor=\color{white},
		breaklines=true,
		breakatwhitespace=false,
		keepspaces=true,
		morekeywords={matlab2tikz},
		showspaces=false, 
		showtabs=false,
		tabsize=2,
		rulecolor=\color{black},
		frame=single,
		keywordstyle={\scriptsize\color{blue}},
		morekeywords=[2]{1}, 
		keywordstyle=[2]{\scriptsize\color{black}},
		identifierstyle={\scriptsize\color{black}},%
		stringstyle={\scriptsize\color{mylilas}},
		commentstyle={\scriptsize\color{mygreen}},
		showstringspaces=false,
		numbers=left,%
		numberstyle={\scriptsize\color{black}},
		numbersep=10pt, 
		emph=[1]{for,end,break},emphstyle=[1]\scriptsize\color{blue}, 
	}
	%\renewcommand{\lstlistingname}{Alg.}

\begin{document}


\maketitle


\begin{frame}
    \frametitle{Conteúdo}
    \tableofcontents
\end{frame}

\section{Sobre o curso}


\begin{frame}
    \frametitle{Sobre o curso}
\begin{itemize}[<+->]
    \item Avaliação:
    \begin{itemize}
        \item primeira metade: um teste e uma prova.
        \item segunda metade: um trabalho e uma prova.
    \end{itemize}
    \item Materiais de estudo:
    \begin{itemize}
        \item \href{https://eclass.fgv.br}{Playlist do Yuri}
        \item \href{https://youtube.com/playlist?list=PL49CF3715CB9EF31D}{Playlist do Gil Strang}
        \item \href{https://bit.ly/acervo-emap}{Acervo do DAMA}
        \item \href{https://sb.fgv.br/catalogo/index.asp?codigo_sophia=117373
}{Livro do Strang} (disponível na biblioteca)
        \item \href{https://sb.fgv.br/catalogo/index.asp?codigo_sophia=130643
}{\warning \hspace{3pt} Livro do Elon} (disponível na biblioteca)
    \end{itemize}
    \item Empenho e participação são a chave para o sucesso.
\end{itemize}
\end{frame}
%%%%%%%%%%%%%%%%%%%%%%%%%%%%%%Frame 4
\section{Contexto Histórico}
\subsection{O que é álgebra?}


%%%%%%%%%%%%%%%%%%%%%%%%%%%%%%Frame 5

\begin{frame}
\frametitle{Etimologia e Origens Árabes}

\begin{itemize}[<+->]
    \item Álgebra vem do árabe \RL{الجبر} (al-jabr), e originalmente significava "união das partes quebradas".

    \item Palavra usada no título original de "Livro Compêndio sobre Cálculo por Restauração e Balanceamento", tratado matemático escrito no século IX pelo matemático e astrônomo persa al-Khwarizmi.

    \item No livro acima, al-Khwarizmi apresenta métodos resolutivos de equações de primeiro e segundo grau.

    \item Em suma, podemos entender a palavra como "técnica de reorganização".
\end{itemize}

\end{frame}


\subsection{Duelos}
\begin{frame}{Álgebra na Itália}

\begin{itemize}[<+->]
    \item Na Alta Idade Média, entre os séculos XI e XIII, os métodos árabes tornaram-se conhecidos na Itália por uma tradução do trabalho de al-Khwarizmi em latim feita por Gerardo de Cremona e pelo trabalho de Leonardo de Pisa (o Fibonacci!).

    \item Devido ao desenvolvimento do comércio, novas técnicas de cálculo são descobertas e aperfeiçoadas.

    \item Em 1463, Benedetto de Florença completa seu renomado trabalho "Trattato di praticha d'arismetica", com mais de 500 páginas. Nos livros 13, 14, e 15, Benedetto trata de equações algébricas.

    \item Inicialmente, ele replica o trabalho de al-Khwarizmi com equações de primeiro e segundo grau. Em seguida, Benedetto introduz a noção de cubo e quarta potência.
\end{itemize}
    
\end{frame}

\begin{frame}{A cobiçada fórmula}

\begin{itemize}[<+->]
    \item No século XVI, Scipione del Ferro determina uma fórmula geral para resolver equações de terceiro grau do tipo $x^3+px=q$, com $p,q,x$ positivos.

    \item Em 1535, Tartaglia também chega a essa fórmula, durante um duelo matemático contra Antonio Maria Fiore, pupilo de Scipione.

    \item Já em 1539, a fórmula é revelada a Cardano, sob juramento de que nunca publicaria a descoberta. Cardano, então, extende a fórmula para equações do tipo $x^3=px+q$ e $x^3+q=px$. No processo, é o primeiro a usar números complexos $a+\sqrt{-b}$, mesmo apreensivo sobre essa "invenção".

    \item Pouco depois, Lodovico Ferrari, servo e aprendiz de Cardano, usa a fórmula de Cardano para resolver equações gerais do quarto grau.
\end{itemize}

\end{frame}

\begin{frame}{Um juramento fatal}

\begin{itemize}[<+->]
    \item Apesar da surpreendente descoberta, Cardano e Ferrari não podem publicá-la devido ao juramento de Cardano.

    \item Assim, em 1543, eles vão a Bolonha encontrar-se com Annibale della Nave, genro de Scipione, para discutir se a solução de equações de terceiro grau já era conhecida antes de Tartaglia.

    \item Após as verificações do trabalho póstumo de Scipione, Cardano publica as soluções gerais de equações do terceiro e quarto grau em seu livro "Ars Magna" (1545).

    \item Tartaglia descobre e fica furioso, inclusive publicando o juramento feito por Cardano.
\end{itemize}

\end{frame}

\subsection{O Último Duelo}
\begin{frame}{Uma busca infindável}

\begin{itemize}[<+->]
    \item Nos anos seguintes, matemáticos como Waring, Vandermonde, Lagrange, Malfatti, Ruffini, Cauchy, Gauss e Abel se debruçaram sobre o problema da resolução de equações algébricas.

    \item Em outubro de 1811, nasce Evariste Galois, matemático francês que em seus 20 anos de vida mudou tudo o que se conhecia por álgebra.

    \item Galois provou que era impossível expressar uma fórmula geral de equações do quinto grau por radicais. Os argumentos e as técnicas usadas foram base para a Teoria de Galois.
\end{itemize}

\end{frame}

\begin{frame}{Um trágico fim}

\begin{itemize}[<+->]
    \item As descobertas de Galois, em vida, não foram entendidas e deixadas de lado. Poisson, num relatório sobre o manuscrito de Galois, disse que "para formar uma opinião, seria necessário esperar o autor publicar sua obra integralmente".

    \item Dezesseis meses depois, Galois é morto num duelo.

    \item Na noite anterior, Galois escreve uma carta para seu amigo Auguste Chevalier explicando as ideias fundamentais de sua teoria.
\end{itemize}

\end{frame}
\begin{frame}{Uma nova álgebra}
% Slide 1
\only<1>{\[ a \hspace{5pt} b \]}
% Slide 2
\only<2>
{
    \[ a\cdot b\]
}
% Slide 3
\only<3>
{
    \[ a\cdot b=e\]
}
\end{frame}

\subsection{Mas por que linear?}
\begin{frame}{Linearidade}

\begin{itemize}[<+->]
    \item Entre 1844 e 1862, o matemático alemão Hermann Grassmann publicou uma série de livros chamada "Ausdehnungslehre" (lit. "Teoria da Extensão").

    \item Nesses trabalhos, a partir de certas suposições a mais a partir de estruturas algébricas gerais (entre elas, a linearidade), Grassmann define conceitos como espaço vetorial, dimensão, produto interno, base, transformação linear e determinante.

    \item Desde então, a álgebra linear se mostrou um campo onde ideias criativas e úteis surgem rotineiramente.
\end{itemize}
    
\end{frame}


\begin{frame}{Fontes}

\begin{itemize}
    \item \href{https://sb.fgv.br/catalogo/index.asp?codigo_sophia=226637
}{WAERDEN, B. L. Van der. A history of algebra : from al-Khwarizmi to Emmy Noether (1985)}

    \item \href{https://www.jstor.org/stable/pdf/2320145.pdf}{Fearnley-Sander, Desmond, "Hermann Grassmann and the Creation of Linear Algebra", American Mathematical Monthly 86 (1979), pp. 809–817}
\end{itemize}
    
\end{frame}

\section{O que vamos ver nesse curso}

\begin{frame}{Conteúdo}
    \pause
    \centering Muitas coisas!
\end{frame}

\begin{frame}{Teoremas}
    \begin{block}{Método dos Mínimos Quadrados}
    Permite fazer previsões e inferências sobre grandes bases de dados.
\end{block}\pause

\begin{block}{Teorema Espectral}
    Permite fatorar matrizes simétricas de uma forma simples.
\end{block}\pause

\begin{block}{Fatoração SVD}
    Permite fatorar \emph{qualquer} matriz de uma forma simples.
\end{block}
\end{frame}

\begin{frame}{Aplicações}
    \begin{itemize}[<+->]
        \item Reconhecimento facial.

        \item Resolução numérica de equações diferenciais.

        \item Modelagem de epidemias.

        \item Processamento de imagens.

        \item Google.

        \item Ranking esportivo.
    \end{itemize}
\end{frame}

\begin{frame}{Fim da Apresentação!}

    \centering{\huge Bom curso!}
\end{frame}

\end{document}